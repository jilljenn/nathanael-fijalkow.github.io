\documentclass[svgnames]{beamer}
\mode<presentation>
\usefonttheme{serif}
\usecolortheme{dove}
\useinnertheme{rounded}
\setbeamercolor{item projected}{fg=black}
\setbeamertemplate{navigation symbols}{}

\usepackage[english]{babel}
\usepackage[latin1]{inputenc}
\usepackage{times}
\usepackage{amsthm,amssymb,amsmath,graphicx}
%\usepackage[usenames]{color}
\usepackage{gastex}
\usepackage{framed}
\usepackage{ifthen}
\usepackage{tikz}

\newtheorem{conjecture}{Conjecture}

\newcommand{\set}[1]{\{ #1 \}}
\newcommand{\rk}{\mathrm{rk}}
\newcommand{\lvl}{\mathrm{lvl}}
\newcommand{\Safe}{\mathrm{Safe}}

\newcommand{\Res}{\mathrm{Res}}
\renewcommand{\next}{\nu}
\newcommand{\mem}{\mathrm{mem}}
\newcommand{\M}{\mathcal{M}}
\newcommand{\up}{\mu}
\newcommand{\N}{\mathbb{N}}
\newcommand{\co}{\mathrm{Co}}
\newcommand{\play}{\pi}
\newcommand{\VE}{V_\exists}
\newcommand{\VA}{V_\forall}
\newcommand{\W}{\mathcal{W}}
\newcommand{\A}{\mathcal{A}}
\newcommand{\WE}{\W_{E}}
\newcommand{\WA}{\W_{A}}
\newcommand{\G}{\mathcal{G}}

%%%%%%%%%%%%%%%%%%%%%%%%%%%%%%%%%%%%%%%%%%%%%%%%%%%%%%%%%%%%%%%%%%%%%%%%%%%%%%%
%%%%%%%%%%%%%%%%%%%% A non-original creation by Nathanaël Fijalkow and myself %

\setbeamertemplate{frametitle}{%
  \vskip-2pt%
  \begin{beamercolorbox}[rightskip=2cm,leftskip=1em,dp=1ex,wd=12.8cm]{frametitle}%
    \vskip2pt%
    \usebeamercolor{frametitle}%
    \begin{tikzpicture}[scale=1]%
      \useasboundingbox (0,0) rectangle (0,0); %(-1,-1) rectangle (1,1);%
      \ifthenelse{\insertframenumber<\inserttotalframenumber}%
      { % uncomplete tart

        \pgfmathsetmacro{\aimangle}{90-(\insertframenumber*360/\inserttotalframenumber)}
        \fill [fill=frametitle.fg,thin, color=gray!50,draw=black] (11.8,.2) -- (11.8,.6) arc (90:\aimangle:0.4) -- cycle;%

      }{ % the full tart
        \fill[fill=frametitle.fg,thin, color=gray!50,draw=black] (11.8,0.2) circle (.4);%
      }%
      \fill[fill=frametitle.fg,thin, color=white,draw=black] (11.8,0.2) circle (.3);%
      \node at (11.8, .2) [black,circle]{\normalsize\insertframenumber};

    \end{tikzpicture}
    \insertframetitle%
    \vskip2pt%
  \end{beamercolorbox}%
}
%%%%%%%%%%%%%%%%%%%%%%%%%%%%%%%%%%%%%%%%%%%%%%%%%%%%%%%%%%%%%%%%%%%%%%%%%%%%%%%

\title{Playing Safe}
\subtitle{FSTTCS'2014}
\author{Thomas Colcombet, \underline{Nathana\"el Fijalkow} and Florian Horn}
\institute{Delhi, India}
\date{December 16th, 2014}

\AtBeginSection[]
{
  \begin{frame}<beamer>{Outline}
    \tableofcontents[currentsection,currentsubsection]
  \end{frame}
}

\begin{document}

\begin{frame}
  \titlepage
\end{frame}

\begin{frame}{Games}

\begin{center}
\begin{picture}(100,50)(0,0)
	\gasset{Nw=8,Nh=8}
	
  	\node(Eve)(60,35){}
	\put(66,34){controlled by Eve}
  	\rpnode[polyangle=45](Adam)(60,25)(4,5){}
	\put(66,24){controlled by Adam}

  	\rpnode[polyangle=45](1)(30,15)(4,5){}
  	\node(2)(20,30){}
  	\node(3)(40,30){}
  	\rpnode[Nmarks=i,iangle=180,polyangle=45](4)(0,30)(4,5){}
  	\rpnode[polyangle=45](5)(10,50)(4,5){}
  	\node(6)(10,10){}
  	\rpnode[polyangle=45](7)(30,50)(4,5){}

  	\drawedge(1,2){}
  	\drawedge[curvedepth=3](1,3){}
	\drawloop[loopangle=90](2){}
  	\drawedge(2,3){}
  	\drawedge(2,6){}
  	\drawedge[curvedepth=5](3,1){}
  	\drawedge(4,2){}
  	\drawedge[curvedepth=5](4,5){}
  	\drawedge[curvedepth=5](5,4){}
	\drawloop[loopangle=-90](6){}
  	\drawedge(7,5){}

\only<3>{\drawedge[AHLength=3,AHlength=4,linecolor=red,linewidth=0.7](4,2){}}
\only<5>{\drawedge[AHLength=3,AHlength=4,linecolor=red,linewidth=0.7](2,6){}}

\only<2,3>{\node[fillcolor=magenta,Nw=4,Nh=4](pebble)(0,30){}} %pebble on 4
\only<4,5>{\node[fillcolor=magenta,Nw=4,Nh=4](pebble)(20,30){}} %pebble on 2
\only<6>{\node[fillcolor=magenta,Nw=4,Nh=4](pebble)(10,10){}} %pebble on 6

\only<7>{
  	\drawedge(1,2){$a$}
  	\drawedge[curvedepth=3](1,3){$b$}
	\drawloop[loopangle=90](2){$b$}
  	\drawedge(2,3){$a$}
  	\drawedge(2,6){$a$}
  	\drawedge[curvedepth=5](3,1){$a$}
  	\drawedge(4,2){$b$}
  	\drawedge[curvedepth=5](4,5){$b$}
  	\drawedge[curvedepth=5](5,4){$b$}
	\drawloop[loopangle=-90](6){$a$}
  	\drawedge(7,5){$b$}

	\put(60,10){\begin{huge}$W \subseteq \set{a,b}^\omega$\end{huge}}
}
\end{picture}
\end{center}
\end{frame}

\begin{frame}{Strategy (for Eve)}
General form 
$$\sigma : V^+ \rightarrow V$$
\pause

Positional or memoryless
$$\sigma : V \rightarrow V$$
\pause

\vskip1em
Finite-memory 
$$\left \{
\begin{array}{c}
\sigma : V \times M \rightarrow V \\
\mu : M \times E \rightarrow M
\end{array}
\right.$$
\end{frame}

\begin{frame}{How much memory is needed to win?}

Let $W \subseteq A^\omega$, compute:
$$\mem(W)\ \doteq\ \sup_{\G = (\A,W) \textrm{ game}}\ 
\inf_{\begin{subarray}{c}\sigma \textrm{ winning}\\ \textrm{strategy}\end{subarray}}\ \mem(\sigma)\ .$$

\pause
\vskip2em
Equivalently:
\begin{itemize}
	\item \textit{upper bound:} for all games $\G = (\A,W)$, if Eve has a winning strategy,
	then she has a winning strategy using at most $\mem(W)$ memory states,
	\item \textit{lower bound:} there exists a game $\G = (\A,W)$ where Eve has a winning strategy,
	but no winning strategy using less than $\mem(W)$ memory states.
\end{itemize}
\end{frame}

\begin{frame}{A Simple Example}

\begin{center}
\begin{picture}(100,50)(0,0)
	\gasset{Nw=7,Nh=7}

	\drawline[AHnb=0](55,60)(55,15)
	\put(20,55){\large{arena}}
	\put(68,55){\large{winning condition}}
	
  	\node[Nmarks=i,iangle=180](2)(20,30){}
  	\node(3)(40,40){}

  	\drawedge(2,3){$b$}
	\drawloop[loopangle=-45](2){$a$}
	\drawloop[loopangle=45](3){}
	
	\put(70,45){\huge{$W = a^+ \cdot b$}}

	\gasset{Nw=6,Nh=6}

  	\node[Nmarks=i,iangle=180](eps)(70,25){}
  	\node(a)(85,25){}
  	\node[Nmarks=f](ab)(100,25){$F$}

  	\drawedge(eps,a){$a$}
  	\drawedge(a,ab){$b$}
	\drawloop[loopangle=90](a){$a$}

\only<3,4,5>{

  	\node[Nmarks=i,iangle=180](eps)(25,0){$\varepsilon$}
	\put(20,-6){play $a$}

  	\node(a)(40,0){$a$}
	\put(35,-6){play $b$}

	\node[linecolor=White](invisible)(55,0){}

  	\drawedge(eps,a){$a$}
  	\drawedge(a,invisible){$b$}

	\put(10,10){A winning strategy for Eve uses two memory states.}
	\put(55,-1){a memory structure}
}

\only<2,3>{
	\node[fillcolor=magenta,Nw=4,Nh=4](pebble)(20,30){}
}

\only<3>{
	\node[linecolor=red,linewidth=0.5](pebble)(25,0){}
}

\only<4>{
	\drawloop[loopangle=-45,AHLength=3,AHlength=4,linecolor=red,linewidth=0.5](2){}
	\drawedge[AHLength=3,AHlength=4,linecolor=red,linewidth=0.5](eps,a){}
} 

\only<5>{
	\drawedge[AHLength=3,AHlength=4,linecolor=red,linewidth=0.5](2,3){}
	\node[fillcolor=magenta,Nw=4,Nh=4](pebble)(20,30){}
	\node[linecolor=red,linewidth=0.5](pebble)(40,0){}
} 

\end{picture}
\end{center}
\end{frame}

\begin{frame}{What is Known about Computing $\mem(W)$}
\begin{theorem}[Dziembowski, Jurdzi\'nski, Walukiewicz, 1997]
For $W$ a boolean combination of ``infinitely many $a \in A$'',
$\mem(W)$ is computable (and characterized through the Zielonka tree).
\end{theorem}

\begin{theorem}[Kopczy\'nski, 2007]
For $W$ which is $\omega$-regular,
$\mem_\mathrm{chromatic}(W)$ is computable.
\end{theorem}

\pause
\vskip2em
\begin{conjecture}[Kopczy\'nski, 2008]
For $W$ which is $\omega$-regular, $\mem_\mathrm{chromatic}(W) = \mem(W)$.
\end{conjecture}
\end{frame}

\begin{frame}{Our Results}
$\Res(W)$ is the set of residuals of $W$: for $u \in \Sigma^*$, 
$$u^{-1} W = \set{v \mid u \cdot v \in W}\ .$$

\begin{framed}
\begin{theorem}[Colcombet, F., Horn]
\begin{center}
For all safety conditions $W$,\\
$\mem(W)$ is the width of $(\Res(W),\subseteq)$,\\
\textit{i.e.} the size of the largest antichain.
\end{center}
\end{theorem}
\end{framed}

\pause
\vskip2em
\begin{itemize}
	\item We make no regularity assumption!
	\item This holds for infinite arenas of finite degree.
\end{itemize}
\end{frame}

\begin{frame}{An upper bound}
The memory structure $\M_W$ uses $\Res(W)$ as set of memory states, and:
\begin{itemize}
	\item the initial memory state is $\varepsilon^{-1} W = W$,
	\item each time a letter $a$ is read from $u^{-1} W$, 
the memory is updated to $(u \cdot a)^{-1} W$.
\end{itemize}

\pause

\begin{lemma}[An upper bound]
For all games $\G = (\A,W)$,
Eve has a winning strategy using $\M_W$.
\end{lemma}
\end{frame}

\begin{frame}{Another example}
\begin{center}
``read at most ten consecutive $a$'s, and then an $b$''.
\end{center}
$$W = a + b\cdot a + bb\cdot a + \ldots + b^{10} \cdot a.$$

\pause
\vskip1em
In every game with condition $W$, Eve wins without memory.

\pause
\vskip1em
$\looparrowright$ This shows that the memory structure $\M_W$ is not optimal.
\end{frame}

\begin{frame}{Order left quotients inclusion-wise}
If Eve wins from $(q,u^{-1} L)$ and $u^{-1} L \subseteq v^{-1} L$, 
then she wins from $(q,v^{-1} L)$ \pause \textit{using the same strategy}.

\pause
\vskip1em
Intuition: whenever in $(q,v^{-1} L)$, play as from $(q,u^{-1} L)$, where $u^{-1} L$ is minimally winning from $q$.

\pause
\vskip1em
Problems:
\begin{itemize}
	\item there may not exist minimally winning left quotients!
	\item winning or losing depends on the current position, which makes updating the memory state not trivial.
\end{itemize}
\end{frame}

\begin{frame}{Our results}
\begin{framed}
\begin{theorem}[Colcombet, F., Horn]
\begin{center}
For all safety conditions $W$,\\
$\mem(W)$ is the width of $(\Res(W),\subseteq)$,\\
\textit{i.e.} the size of the largest antichain.
\end{center}
\end{theorem}
\end{framed}

\pause
\vskip2em
\begin{itemize}
	\item Fails for infinite arenas of infinite degree.
	\item Unifies several results from the literature: boundedness condition, energy condition, generalized reachability.
\end{itemize}
\end{frame}

\begin{frame}{The end}

Thanks!

\end{frame}
\end{document}
