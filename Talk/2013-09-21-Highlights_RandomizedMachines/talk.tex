\documentclass[svgnames]{beamer}
\mode<presentation>
\usefonttheme{serif}
\usecolortheme{dove}
\useinnertheme{rounded}
\setbeamercolor{item projected}{fg=black}
\setbeamertemplate{navigation symbols}{}

\usepackage[english]{babel}
\usepackage[latin1]{inputenc}
\usepackage{times}
\usepackage{amsmath}
\usepackage{amsfonts}
\usepackage{amssymb}
\usepackage{amsthm}
\usepackage{graphics}
\usepackage{multicol}
\usepackage{framed}
\usepackage{ulem}
\usepackage{ifthen}
\usepackage{tikz}
\usepackage{gastex}
\usepackage{ulem}
\usepackage{booktabs}
\usepackage{tikz}
\usetikzlibrary{shapes}
\usepackage{subfigure}
\usepackage{wasysym}
\usepackage{xspace}
\usepackage{bbm}
\usepackage{yfonts}
\usepackage[bbgreekl]{mathbbol}
\DeclareMathAlphabet{\mathpzc}{OT1}{pzc}{m}{it}
\usepackage{nicefrac}

\newcommand{\prob}[1]{\mathbb{P}_{#1}}
\newcommand{\A}{\mathcal{A}}

%%%%%%%%%%%%%%%%%%%%%%%%%%%%%%%%%%%%%%%%%%%%%%%%%%%%%%%%%%%%%%%%%%%%%%%%%%%%%%%
%%%%%%%%%%%%%%%%%%%% A non-original creation by Nathanaël Fijalkow and myself %

\setbeamertemplate{frametitle}{%
  \vskip-2pt%
  \begin{beamercolorbox}[rightskip=2cm,leftskip=1em,dp=1ex,wd=12.8cm]{frametitle}%
    \vskip2pt%
    \usebeamercolor{frametitle}%
    \begin{tikzpicture}[scale=1]%
      \useasboundingbox (0,0) rectangle (0,0); %(-1,-1) rectangle (1,1);%
      \ifthenelse{\insertframenumber<\inserttotalframenumber}%
      { % uncomplete tart

        \pgfmathsetmacro{\aimangle}{90-(\insertframenumber*360/\inserttotalframenumber)}
        \fill [fill=frametitle.fg,thin, color=gray!50,draw=black] (11.8,.2) -- (11.8,.6) arc (90:\aimangle:0.4) -- cycle;%

      }{ % the full tart
        \fill[fill=frametitle.fg,thin, color=gray!50,draw=black] (11.8,0.2) circle (.4);%
      }%
      \fill[fill=frametitle.fg,thin, color=white,draw=black] (11.8,0.2) circle (.3);%
      \node at (11.8, .2) [black,circle]{\normalsize\insertframenumber};

    \end{tikzpicture}
    \insertframetitle%
    \vskip2pt%
  \end{beamercolorbox}%
}
%%%%%%%%%%%%%%%%%%%%%%%%%%%%%%%%%%%%%%%%%%%%%%%%%%%%%%%%%%%%%%%%%%%%%%%%%%%%%%%


\setbeamertemplate{blocks}[rounded]%
\setbeamercolor{block title}{bg=normal text.bg!90!black}
\setbeamercolor{block body}{bg=normal text.bg!95!black}

\begin{document}

\addtocounter{framenumber}{-1}

\date{Highlights, September 21st, 2013}

\author{\underline{Nathana\"el Fijalkow} \and Hugo Gimbert \and\\ Florian Horn \and Youssouf Oualhadj}

\title{We tried and we tried,\\
and we applied and implied,\\
and still probabilistic automata\\
we could not decide!}

\begin{frame}
\maketitle
\end{frame}

\tikzset{
minimum height=.666cm,minimum width=.666cm,circle
}

%%%%%%%%%%%%%%%%%%%%%%%%%%%%%%%%%%%%%%%%%%%%%%%%%%%%%%%%%%%%%%%%%%%%%%%
\begin{frame}{Probabilistic automata}

\begin{figure}
\begin{center}
\begin{picture}(20,40)(0,-5)
	\gasset{Nw=6,Nh=6}

  	\node[Nmarks=i,iangle=0](L1)(15,15){$1$}
  	\node(L2)(0,30){$2$}
  	\node[Nmarks=r](L3)(0,0){$3$}

  	\drawedge(L1,L2){$b$}
  	\drawedge[curvedepth=-5,ELside=r](L1,L3){$a,0.4$}
  	\drawedge[curvedepth=-5,ELside=r](L3,L1){$b$}
	\drawloop(L1){$a,0.6$}
	\drawloop[loopangle=135](L2){$a,b$}
	\drawloop[loopangle=215](L3){$a$}
\end{picture}
\end{center}
\end{figure}

\pause
\begin{block}{Early results}
	\begin{itemize}
		\item (Paz, 71) The emptiness problem is undecidable;
		\item (Rabin, 69) If $\lambda$ is isolated, $L_\lambda$ is regular;
		\item (Bertoni, 74) The isolation problem is undecidable;
		\item (Condon-Lipton, 89) The approximation problem is undecidable.
	\end{itemize}
\end{block}
\end{frame}


\begin{frame}{Recent results}

The isolation problem for $\lambda = 1$ is
\begin{center}
``are there words accepted with probability arbitrarily close to $1$''.
\end{center}

\pause
\begin{block}{Recent results}
	\begin{itemize}
		\item (Gimbert and Oualhadj, 2009) The value $1$ problem is undecidable;
		\item Decidable classes: $\sharp$-acyclic $\subsetneq$ structurally simple $\subsetneq$ leaktight.
	\end{itemize}
\end{block}
\end{frame}


\begin{frame}{Undecidability results: towards fuzziness}

\begin{itemize}
	\item (Paz, 71) The emptiness problem:
	\[
	\exists w,\ \prob{\A}(w) \ge \frac{1}{2}
	\]
	\item (Bertoni, 74) The isolation problem:
	\[
	\forall \varepsilon, \exists w,\ \frac{1}{2} - \varepsilon \le \prob{\A}(w) \le \frac{1}{2} + \varepsilon
	\]	
	\item (Condon-Lipton, 89) The approximation problem:
	\[
	\exists w, \prob{\A}(w) \ge \frac{2}{3} \qquad \vee \qquad \forall w, \prob{\A}(w) \le \frac{1}{3}
	\]
	\item (Gimbert-Oualhadj, 2009) The value $1$ problem:
	\[
	\forall \varepsilon, \exists w,\ \prob{\A}(w) \ge 1 - \varepsilon
	\]
\end{itemize}
\end{frame}

\begin{frame}{Research questions}
\begin{block}{}
	\begin{itemize}
		\item What does the saturation algorithm compute?
		\item What is decidable for probabilistic automata? 

		\textcolor{red}{		
		How much fuzziness is required to get decidability?
		}
	\end{itemize}
\end{block}

Numberless probabilistic automata: is the value $1$ problem decidable?
\end{frame}

\begin{frame}{Yet still PA we could not decide!}

Randomized machines are deterministic automata with only \textit{one} random coin.
\vskip1em

\pause
\begin{theorem}
The emptiness problem, the isolation problem and the approximation problems
are \textit{all} undecidable for randomized machines.
\end{theorem}

\pause
\begin{corollary}
The value $1$ problem for numberless probabilistic automata are undecidable.
\end{corollary}

\pause
Conclusion: the saturation algorithm is useless.
\end{frame}

\end{document}
